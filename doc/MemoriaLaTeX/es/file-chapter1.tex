\tocchapter{Primer cap�tulo}

Este es el primer cap\'{\i}tulo.

\section{Primera secci�n}

Esta es la primera secci\'{o}n del primer cap\'{\i}tulo.

Ver\'{a}s que los p\'{a}rrafos comienzan indentados, como por ejemplo este mismo p\'{a}rrafo est\'{a} indentado, como puedes ver.

\noindent Puedes evitar la indentaci\'{o}n con \verb#\noindent#. Por ejemplo, esta l\'{\i}nea.

\noindent Ejemplo de lista de \'{\i}tems:

\begin{itemize}
 \item Esta es una cita de ejemplo\cite{yo:07} (del fichero de bibliograf\'{a} \textit{file-bibliography.bib}).
 \item Esta es otra cita\cite{yo:08} (del fichero \textit{file-bibliography2.bib}).
 \item Ver Tabla~\ref{tab}.
 \item Ver Figura~\ref{fig}.
 \item Ver Eq.~\ref{eq}
\end{itemize}

\begin{table}[hbt]
 \begin{center}
 \begin{tabular}{cc}
  a & b \\
  c & d 
 \end{tabular}
 \end{center}
 \caption{Esto es una tabla.}
 \label{tab}
\end{table}

\begin{figure}[hbt]
 \begin{center}
  % Descomentar la l�nea siguiente, cambiando FILENAME por el nombre del fichero (E)PS
  %\includegraphics[width=1.0\cw]{FILENAME.eps}
 \end{center}
 \caption{Esta figura est\'{a} vac\'{\i}a.}
 \label{fig}
\end{figure}

\begin{equation}
  1 + 1 = 7
 \label{eq}
\end{equation}
